\chapter{Problem analysis}

Stereometry is an essential part of mathematics education.
It develops a student's ability to understand and reason about three-dimensional space.
Knowledge of stereometry is important not only for further studies in mathematics, but also for fields such as physics, engineering, architecture, and computer science.
In those fields, spatial thinking and geometric reasoning are frequently required.
Through the study of three-dimensional solids, students learn to work with concepts such as volume, surface area, and spatial relationships between objects.

Despite its importance, stereometry is often regarded as a difficult subject by students.
One of the main challenges lies in the need to imagine three-dimensional objects based on two-dimensional representations.
Two-dimensional representations include drawings on a board or in a textbook.
Static images and traditional teaching methods may not sufficiently convey the spatial structure of solids.
This may lead to poor understanding and memorization without true comprehension.

An interactive application capable of visualizing three-dimensional solids can help with these difficulties.
By allowing users to observe objects from different perspectives, inspect their components, and manipulate their parameters, such an application supports the development of spatial imagination.
Visual feedback enables students to better understand how changes in dimensions or structure affect the properties of a solid.
This strengthens the connection between geometric theory and its practical interpretation.

\section{Mathematical description of selected solids}

\subsection{Prism}

A prism is a polyhedron consisting of two congruent \emph{n}-sided polygon bases and \emph{n} joining faces. Every joining face is a parallelogram.
A prism is called after its base, e.g. a prism with a hexagonal base is called a hexagonal prism.
An \emph{n}-gonal prism has $n + 2$ faces, $3n$ edges, and $2n$ vertices.

\subsubsection*{Special cases}

An \emph{oblique prism} is a prism whose joining faces are not perpendicular to the bases.
A prism whose whose joining faces are perpendicular to the bases is a \emph{right prism}.

A right prism with a rectangular base is also called a \emph{cuboid} (see figure \ref{prism}).
A right prism with a square base is also called a \emph{square cuboid}.
A square cuboid whose height is equal to the side length of its base is a \emph{cube}.

\begin{figure}[H]
    \centering
    \includegraphics[width=0.5\textwidth]{graf/prism}
    \caption{A right prism with a rectangular base.}
    \label{prism}
\end{figure}

\subsubsection*{Properties}

\paragraph{Volume}

The volume of a prism is the product of the base area and the height.
The formula for the volume is:
\begin{equation}
    V = B h,
\end{equation}
where \emph{B} is the base area and \emph{h} is the height.

\paragraph{Surface area}

The surface area of a prism is the sum of all its faces' areas.
The formula for the surface area is:
\begin{equation}
    A= 2B + P h,
\end{equation}
where \emph{B} is the base area, \emph{P} the perimeter, and \emph{h} the height.

\subsection{Pyramid}

A pyramid is a polyhedron consisiting of one polygonal base face connected to a point, called the apex.
Each of the base's edges is connected to the apex, forming a triangle, called a lateral face.
A pyramid with an \emph{n}-gonal base has $n + 1$ faces, $2n$ edges, and $n + 1$ vertices.

\subsubsection*{Special cases}

A pyramid is called a \emph{regular pyramid} when its base is a regular polygon.
The definition of a \emph{right pyramid} varies between sources.
Some sources define it as a regular pyramid whose height falls on the center of the base \cite{gottwald2012vnr}.
According to other sources, a right pyramid is a pyramid, whose height falls on the center of a circle circumscribed about the pyramid's base \cite{polya1990mathematics}.
A right pyramid with a hexagonal base can be seen in figure \ref{pyramid}.
A pyramid with a triangular base is also called a \emph{tetrahedron}.

\begin{figure}[H]
    \centering
    \includegraphics[width=0.5\textwidth]{graf/pyramid}
    \caption{A right pyramid with a hexagonal base.}
    \label{pyramid}
\end{figure}

\subsubsection*{Properties}

\paragraph{Volume}

The volume of a pyramid is one third of the product of the base area and the height.
The formula for the volume is:

\begin{equation}
    V = \frac{1}{3} B h,
\end{equation}
where \emph{B} is the base area and \emph{h} is the height \cite{lang_murrow_geometry}.

\paragraph{Surface area}

The surface area of a pyramid is the sum of the area of the base and the area of the lateral faces.
For a regular right pyramid with an \emph{n}-gonal base and height \emph{h}, the surface area can be calculated as follows:

Each regular \emph{n}-gon can be divided into \emph{n} congruent triangles.
The base of the triangle is the side \emph{s} of the pyramid.
The triangle's height $h_{B}$ is calculated as:
\begin{equation}
    h_{B} = \frac{1}{2} s \tan (\frac{n-2}{n}90^{\circ}).
\end{equation}
Then, the polygon base area can be calculated by multiplying the area of one triangle by the number of sides:
\begin{equation}
\begin{split}
    A_{B} & = \frac{s h_{B}}{2} n.
\end{split}
\end{equation}
Height of lateral triangles can be obtained by using the Pythagorean theorem:
\begin{equation}
    h_{L} = \sqrt{{h_{B}}^2 + h^2}.
\end{equation}
The lateral area is calculated in a similar way to the base area:
\begin{equation}
\begin{split}
    A_{L} & = \frac{s h_{L}}{2} n.
\end{split}
\end{equation}
Finally, the total area is the sum of the base area and the lateral area:
\begin{equation}
\begin{split}
    A  & = A_{B} + A_{L}.
\end{split}
\end{equation}

\newpage

\section{Existing solutions}

% https://www.geogebra.org/3d
\subsection{GeoGebra 3D Calculator}

The 3D Calculator\footnote{\url{https://www.geogebra.org/3d}} by GeoGebra is an extensive application which provides many tools for exploring 3D geometry.
It allows the user to create and manipulate solids, points, lines, and polygons.
It also provides tools for measuring angles, distances, areas, and volumes.
The 3D Calculator can be of great help when exploring stereometry, but it does not provide didactic features.
It is not suited for structured, guided learning. The measuring tools allow the user to calculate certain characteristics but do not teach how to do so.
GeoGebra 3D Calculator does not provide support for keyboard control, the mouse being the only allowed input method.
The application is only available on web and mobile platforms with no downloadable version for desktop computers.
The user interface of the GeoGebra 3D Calculator is visible in figure \ref{geogebra3d}.

\begin{figure}[H]
    \centering
    \includegraphics[width=\textwidth]{graf/geogebra_3d}
    \caption{GeoGebra 3D Calculator.}
    \label{geogebra3d}
\end{figure}

\subsection{Cabri 3D and Cabri Express}

Cabri 3D\footnote{\url{https://cabri.com/en/enterprise/cabri-3d/index.html}} and Cabri Express\footnote{\url{https://cabri.com/fr/enterprise/cabri-express/}} are 2 applications for learning three-dimensional geometry developed by Cabri.
Both applications are designed with teaching in mind.
Cabri products are closed-source, which means that the software will no longer be able to receive updates after the original creators stop developing them.

% https://cabri.com/en/instructor/cabri-3d/
\subsubsection*{Cabri 3D}

Cabri 3D is an application for the Windows and Mac OS operating systems.
It is commercial software which requires the purchase of a license.
The application is used primarily to create documents consisting of text areas and 3D views.
This makes it suitable to be used by teachers to prepare lesson materials, which can be printed on paper.
It is not suited to by used by students to explore 3D geometry or solve tasks.
The user inteface has an outdated look and has some bugs when running on a system with multiple displays.
When running on a system with two displays, the \emph{Tool Help} window produces visual artifacts on the other display when it is moved.
The user interface of Cabri 3D is visible in figure \ref{cabri3d}.

\begin{figure}[H]
    \centering
    \includegraphics[width=\textwidth]{graf/cabri_3d}
    \caption{Cabri 3D.}
    \label{cabri3d}
\end{figure}

% https://cabri.com/en/instructor/cabri-express/
\subsubsection*{Cabri Express}

Cabri Express is a general mathematical application and features tools not limited to 3D geometry.
It has tools related to stereometry but it is not specialized in this field.
This, combined with it being closed-source introduces a problem.
If one doesn't find the desired 3D tool in Cabri Express, it may never be added by the developers.
Learning how to use the applications may be problematic, as most of the official tutorials for the applications are in French.
Moreover, there are few unofficial learning resources in other languages.
The user interface of Cabri Express is visible in figure \ref{cabriexpress}.

\begin{figure}[H]
    \centering
    \includegraphics[width=\textwidth]{graf/cabri_express}
    \caption{Cabri Express.}
    \label{cabriexpress}
\end{figure}

\subsection{Shapes 3D Geometry Learning \& Drawing}

Shapes 3D Geometry Learning\footnote{\url{https://shapes.learnteachexplore.com/shapes-3d-geometry-learning/}} and Shapes 3D Geometry Drawing\footnote{\url{https://shapes.learnteachexplore.com/shapes-3d-geometry-drawing/}} are two similar applications whose main feature is viewing solids.
Both applications are available in a subscription model.
One may also buy a perpetual license for the Shapes 3D Geometry Drawing application on the iOS platform.
A free demo accessible from the browser exists for both applications.
Unfortunately, the author of the thesis was unable to run the Shapes 3D Geometry Drawing demo due to an error present on the website.

% https://shapes.learnteachexplore.com/shapes-3d-geometry-learning/
\subsubsection*{Shapes 3D Geometry Learning}

Shapes 3D Geometry Learning is available on all major desktop and mobile operating systems.
The target group of the application is students of grades 4-6.
The application allows the user to select from a set of predefined solids.
When a solid is selected, the user can view its edges, faces, and vertices.
The solid can be unfolded it into a net. The user may also build the net themself, using the solid's faces.
The net can be printed and subsequently cut out of the paper, to create a paper net.
The paper net can be glued to obtain a paper version of the solid displayed in the application.
The user interface of Shapes 3D Geometry Learning is visible in figure \ref{shapeslearning}.

\begin{figure}[H]
    \centering
    \includegraphics[width=\textwidth]{graf/shapes_3d_learning}
    \caption{Shapes 3D Geometry Learning.}
    \label{shapeslearning}
\end{figure}

% https://shapes.learnteachexplore.com/shapes-3d-geometry-drawing/
\subsubsection*{Shapes 3D Geometry Drawing}

Shapes 3D Geometry Drawing is available only on the iOS operating system.
The target group of the application is students of grades 7-8.
It has a user interface and application structure similar to Shapes 3D Geometry Learning.
The main differences is the tools available when a solid is selected.
Shapes 3D Geometry Drawing allows the user to place line segments and cross sections between points on the solid's faces.
The error that appears when trying to run Shapes 3D Geometry Drawing is visible in figure \ref{shapesdrawing}.

\begin{figure}[H]
    \centering
    \includegraphics[width=\textwidth]{graf/shapes_3d_drawing}
    \caption{Shapes 3D Geometry Drawing. An error occurs when trying to run the demo.}
    \label{shapesdrawing}
\end{figure}

\subsubsection*{Summary of Shapes 3D applications}

Both applications contain many features which help visualize and explore 3D geometry.
The applications lack features related to calculation of solids' characteristics like area and volume.
They also do not allow the user to create their own solid and only give access to predefined solids.
The predefined solids cannot be manipulated by the user, e.g the scale of the solid cannot be changed.

% \begin{itemize}
% \item  problem analysis
% \item state of the art, problem statement
% \item  literature research (all sources in the thesis have to be referenced \cite{bib:article,bib:book,bib:conference,bib:internet})
% \item description of existing solutions (also scientific ones, if the problem is scientifically researched), algorithms,  location of the thesis in the scientific domain
% \end{itemize}

%\begin{Definition}\label{def:definition}
% body of the definitions
%\end{Definition}
%
%\begin{Theorem}[optional name]\label{t:theorem}
%body of the theorem
%\end{Theorem}
%
%\begin{Example}[optional name]\label{ex:example}
%body of the example
%\end{Example}

%%%%%%%%%%%%%%%%%%%%%%%%


