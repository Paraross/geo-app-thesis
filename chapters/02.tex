\chapter{Problem analysis}

\section{Existing solutions}

% https://www.geogebra.org/3d
\subsection{GeoGebra 3D Calculator}

The 3D Calculator \cite{geogebra3d} by GeoGebra is an extensive application which provides many tools for exploring 3D geometry.
It allows the user to create and manipulate solids, points, lines and polygons.
It also provides tools for measuring angles, distances, areas and volumes.
The 3D Calculator can be of great help when exploring stereometry, but it does not provide didactic features.
It is not suited for structured, guided learning. The measuring tools allow the user to calculate certain characteristics but do not teach how to do so.
GeoGebra 3D Calculator does not provide support for keyboard control, the mouse being the only allowed input method.
The application is only available on web and mobile platforms with no downloadable version for desktop computers.

\begin{figure}
    \centering
    \includegraphics[width=\textwidth]{graf/geogebra_3d}
    \caption{GeoGebra 3D Calculator.}
\end{figure}

\subsection{Cabri 3D and Cabri Express}

Cabri 3D \cite{cabri3d} and Cabri Express \cite{cabriexpress} are 2 applications for learning three-dimensional geometry developed by Cabri.
Both applications are designed with teaching in mind.
Cabri products are closed-source, which means that the software will no longer be able to receive updates after the original creators stop developing them.

% https://cabri.com/en/instructor/cabri-3d/
\subsubsection*{Cabri 3D}

Cabri 3D is an application for the Windows and Mac OS operating systems.
It is commencial software which requires the purchase of a license.
The application is used primarily to create documents consisting of text areas and 3D views.
This makes it suitable to be used by teachers to prepare lesson materials, which can be printed on paper.
It is not suited to by used by students to explore 3D geometry or solve tasks.
The user inteface has an outdated look and has some bugs when running on a system with multiple displays.
When running on a system with two displays, the \emph{Tool Help} window produces visual artifacts on the other display when it is moved.

\begin{figure}
    \centering
    \includegraphics[width=\textwidth]{graf/cabri_3d}
    \caption{Cabri 3D.}
\end{figure}

% https://cabri.com/en/instructor/cabri-express/
\subsubsection*{Cabri Express}

Cabri Express is a general mathematical application and features tools not limited to 3D geometry.
It has tools related to stereometry but it is not specialized in this field.
This, combined with it being closed-source introduces a problem.
If one doesn't find the desired 3D tool in Cabri Express, it may never be added by the developers.
Learning how to use the applications may be problematic, as most of the official tutorials for the applications are in French.
Moreover, there are few unofficial learning resources in other languages.

\begin{figure}
    \centering
    \includegraphics[width=\textwidth]{graf/cabri_express}
    \caption{Cabri Express.}
\end{figure}

\subsection{Shapes 3D Geometry Learning \& Drawing}

Shapes 3D Geometry Learning \cite{shapes3dlearning} and Shapes 3D Geometry Drawing \cite{shapes3ddrawing} are two similar applications whose main feature is viewing solids.
Both applications are available in a subscription model.
One may also buy a perpetual license for the Shapes 3D Geometry Drawing application on the iOS platform.
A free demo accessible from the browser exists for both applications.
Unfortunately, the author of the thesis was unable to run the Shapes 3D Geometry Drawing demo due to an error present on the website.

% https://shapes.learnteachexplore.com/shapes-3d-geometry-learning/
\subsubsection*{Shapes 3D Geometry Learning}

Shapes 3D Geometry Learning is available on all major desktop and mobile operating systems.
The target group of the application is students of grades 4-6.
The application allows the user to select from a set of predefined solids.
When a solid is selected, the user can view its edges, faces and vertices.
The solid can be unfolded it into a net. The user may also build the net themself, using the solid's faces.
The net can be printed and subsequently cut out of the paper, to create a paper net.
The paper net can be glued to obtain a paper version of the solid displayed in the application.

\begin{figure}
    \centering
    \includegraphics[width=\textwidth]{graf/shapes_3d_learning}
    \caption{Shapes 3D Geometry Learning.}
\end{figure}

% https://shapes.learnteachexplore.com/shapes-3d-geometry-drawing/
\subsubsection*{Shapes 3D Geometry Drawing}

Shapes 3D Geometry Drawing is available only on the iOS operating system.
The target group of the application is students of grades 7-8.
It has a user interface and application structure similar to Shapes 3D Geometry Learning.
The main differences is the tools available when a solid is selected.
Shapes 3D Geometry Drawing allows the user to place line segments and cross sections between points on the solid's faces.

\begin{figure}
    \centering
    \includegraphics[width=\textwidth]{graf/shapes_3d_drawing}
    \caption{Shapes 3D Geometry Drawing. An error occurs when trying to run the demo.}
\end{figure}

\subsubsection*{Summary of Shapes 3D applications}

Both applications contain many features which help visualize and explore 3D geometry.
The applications lack features related to calculation of solids' characteristics like area and volume.
They also do not allow the user to create their own solid and only give access to predefined solids.
The predefined solids cannot be manipulated by the user, e.g the scale of the solid cannot be changed.

% \begin{itemize}
% \item  problem analysis
% \item state of the art, problem statement
% \item  literature research (all sources in the thesis have to be referenced \cite{bib:article,bib:book,bib:conference,bib:internet})
% \item description of existing solutions (also scientific ones, if the problem is scientifically researched), algorithms,  location of the thesis in the scientific domain
% \end{itemize}

%\begin{Definition}\label{def:definition}
% body of the definitions
%\end{Definition}
%
%\begin{Theorem}[optional name]\label{t:theorem}
%body of the theorem
%\end{Theorem}
%
%\begin{Example}[optional name]\label{ex:example}
%body of the example
%\end{Example}

%%%%%%%%%%%%%%%%%%%%%%%%


