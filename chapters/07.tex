\chapter{Conclusions}

\section{Obtained result}

The goal of the project was to create an educational application which made learning stereometry easier.
The finished result achieves this goal with the use of three modules, namely task exploration, task solving and the playground.

\section{Further development}

The application meets the specified functional and non-functional requirements.
Despite that, there is room for improving the existing features and adding new ones.

The keyboard controls for the 3D view currently cannot be modified.
This could be improved by allowing the user to change the controls in the settings.

Another improvement for the playground is related to the polyhedron's vertices.
Currently, the only way to move a vertex is through the vertex list.
Another way of moving a vertex could be accomplished through the 3D view.
Upon clicking a vertex, a control would appear.
The control would allow the user to easily move the vertex on each axis.

The set of tasks is small and could thus be expanded.
New tasks could feature new solids, including non-polyhedrons.

\section{Encountered difficulties}

Even though the used tools are well suited for the project, some issues arose during the development.
The most common issue that appeared multiple times was screen state management.
Some elements, like the polyhedron or task in the 3D view, were not being reset correctly when entering or exiting the screen they were in.
This caused some properties to be set to null and the application to crash.
Godot's debugger helped track the source of these issues.
A potential solution would be to instantiate the screen from scratch when it is entered.
When done this way, care should be taken regarding the performance of such approach.
Another issue was the class structure.
The classes were refactored several times to accomodate for changing requirements.
One example are classes representing solids like cube and prism.
Originally, they were derived from the general \texttt{Figure} class.
Later, a new \texttt{Polyhedron} class was added and the \texttt{Cube} and \texttt{Prism} classes were rewritten to inherit from it.

% \begin{itemize}
% \item achieved results with regard to objectives of the thesis and requirements
% \item path of further development (eg functional extension …)
% \item encountered difficulties and problems
% \end{itemize}

