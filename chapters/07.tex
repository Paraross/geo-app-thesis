\chapter{Conclusions}

\section{Obtained result}

The goal of the project was to create an educational application which made learning stereometry easier.
The finished result achieves this goal with the use of three modules, namely task exploration, task solving, and the playground.

The task exploration module teaches the user how to solve various tasks from the field of stereometry.
It prepares the user for solving the tasks themself in the task solving module.
The ability to change task parameter values helps the user visualize how those changes influence how the solids featured in the task look.

The task solving module improves the user's skills related to calculation of various characteristics of solids.
By dividing each task into steps, the user learns how to divide a bigger problem into smaller parts.
Multiple difficulties allow the user to select the difficulty appropriate for their skill level.
With time, the user may progress to more difficult tasks.

The playground module allows the user to freely explore 3D geometry by playing with polyhedrons.
The user can create a polyhedron from a list of vertices and see the polyhedron in a 3D view.
The properties of the displayed polyhedron can be inspected to help the user better understand the relationship between the specified vertices and the created polyhedron.

\section{Evaluation of used tools and technologies}

The chosen tools played an important role in the successful completion of the project.
The Godot engine provided scene management, user interface creation, and 3D rendering within a single environment.
This made it a suitable platform for developing an educational application with 3D rendering capabilities.

The use of the Neovim text editor improved the development of the application's source code.
Its extensibility and keyboard-oriented controls allowed for fast editing of scripts.
Neovim's integration with version control tools made version control simpler and more efficient.

Version control was managed using Git and the repository was hosted on GitHub.
This enabled systematic tracking of changes, safe experimentation during development, and easy recovery from errors.
Publishing the source code of the application supports future extension of the application.

\section{Further development}

The application meets the specified functional and non-functional requirements.
Despite that, there is room for improving the existing features and adding new ones.

The keyboard controls for the 3D view currently cannot be modified.
This could be improved by allowing the user to change the controls in the settings.

Another improvement for the playground is related to the polyhedron's vertices.
Currently, the only way to move a vertex is through the vertex list.
Another way of moving a vertex could be accomplished through the 3D view.
Upon clicking a vertex, a control would appear.
The control would allow the user to easily move the vertex on each axis.

The set of tasks is small and could thus be expanded.
New tasks could feature new solids, including non-polyhedrons.

\section{Encountered difficulties}

Even though the used tools were well suited for the project, some difficulties were encountered during the development.
An issue that appeared multiple times was screen state management.
Some elements, like the polyhedron or task in the 3D view, were not being reset correctly when entering or leaving the screen they were in.
Care had to be taken to ensure that the state of each screen would be correct after leaving and entering it.
Another issue was the class structure.
Classes were refactored several times to accommodate for changing requirements.
This was especially prevalent in classes related to solids.

\subsection{Acquired knowledge and experience}

The development of the application provided valuable experience in the design and implementation of an interactive three-dimensional application with a graphical user interface.
As this was the first project of this type for the author, particular emphasis was placed on understanding 3D scene management and spatial transformations.

The project also required integrating 3D visualization with user interface elements.
This led to a deeper understanding of event-driven programming and communication between subsystems of the application.

The Godot engine was used in combination with the Neovim text editor, which required configuring and connecting Neovim to Godot's language server.
This contributed to improved understanding of development tools and communication between them.

Overall, the project significantly expanded the author's practical experience in software development, particularly in the areas of 3D graphics, application architecture, and toolchain configuration.

% \begin{itemize}
% \item achieved results with regard to objectives of the thesis and requirements
% \item path of further development (eg functional extension …)
% \item encountered difficulties and problems
% \end{itemize}

