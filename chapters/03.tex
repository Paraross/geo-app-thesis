\chapter{Requirements and tools}

\section{Functional requirements}

\begin{description}
    \item[Visualization of 3D objects] The application will be able to display three-dimensional solids (e.g. cube, prism, sphere, cylinder).
        The camera will allow for rotation and zooming.
    \item[Inspection of solids] When hovering on a solid, its vertices or edges, the object will be highlighted to visually indicate that it is selected.
        Upon clicking a selected object, an informational label will appear.
        The informational label will contain information related to the object, e.g. the position of a vertex or length of an edge.
    \item[Manipulation of solids] The application will allow the user to change the parameters of tasks.
        The changes in parameters will be reflected in the 3D view.
        The application will contain a module which will allow for creation and manipulation of custom polyhedrons.
    \item[Educational support] The application will contain a task module.
        There will be a number of predefined tasks from the field of stereometry with varying degrees of difficulty.
        Each task will be divided into steps. Each step will involve calculation of one portion of the problem described by the task.
    \item[User interface] The application will provide an intuitive graphical user interface.
        The application will support mouse and keyboard interaction.
        User interface elements will include tooltips to help the user understand how the application works.
\end{description}

\section{Non-functional requirements}

\begin{description}
    \item[Usability] The user interface will be straightforward to use and understand.
        It will not be required for the user to be familiar with other 3D software to use the application effectively.
        Language within the application will be simple and use terms consistent with those use in mathematics education.
    \item[Performance] The application will perform well on hardware capable of smoothly running other similar applications.
        There will be no noticeable stutters and delays when interacting with the application.
    \item[Reliability] The application will not crash during runtime and gracefully handle invalid input.
    \item[Correctness] The application will correctly calculate all characteristics of 3D objects.
    \item[Maintainability] The application code will be open-source.
        The application will be structured in a way allowing easy future extension in the form of new tasks and features.
    \item[Security] The application will not require internet access after installation.
        Manual editing of files used by the application will not compromise security.
    \item[Educational suitability] The application will be suited for use in the classroom and at home.
        The provided tasks will be comparable to tasks from a school stereometry curriculum.
\end{description}

\section{Used tools}

\subsection{Godot Engine}

Godot\footnote{\url{https://godotengine.org}} is a cross-platform, free and open-source game engine.
It is designed to create 2D and 3D games mainly for the PC (personal computer), mobile and web platforms.
Despite its main focus being the creation of games, it can be used to develop other types of software.

The engine itself is written mostly in C++.
The primary language used in Godot to create software is GDScript.
Other than GDScript, C\# and C++ are also officially supported.
Aside from the officially supported languages, support for other languages is provided by community extensions.
Some of the community-supported languages are Rust, Swift and Java.

Godot introuduces the concept of \textit{nodes}.
A node is the basic building block of the application.
A node encapsulates a specific functionality related to 2D or 3D graphics, user interface, etc.
Nodes are organized in a tree structure.
Each node can have child nodes.
This allows for the creation of complex behaviors through the composition of simple elements.

A \textit{scene} is a reusable composition of nodes.
Scenes are typically used to represent logical parts of the application, such as user interface screens or 3D objects.
A scene can be instantiated multiple times and can be composed of other scenes.

Through the use of nodes and scenes, Godot enables modularity and maintainability.
This model is well suited for application which require interactive user interfaces and dynamic scene management.

\subsection{Neovim}

\subsection{Git}

% \begin{itemize}
% \item functional and nonfunctional requirements
% \item use cases (UML diagrams)
% \item description of tools
% \item methodology of design and implementation
% \end{itemize} 
