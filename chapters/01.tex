\chapter{Introduction}

\section{Introduction into the problem domain}

With the development of technology, education changes from classical approaches to ones using tools and methods previously unavailable.
As computers become more accessible and widespread, schools introduce more digital ways of teaching students.
One of the goals of educational applications is to aid teachers in conveing knowledge more effectively than using classical methods.
A well-designed educational application should be able to provide all the tools necessary to teach students a certain subject.
It may also be installed on the student's personal computer, allowing them to study and practice efficiently even outside of school.

The problem domain of the thesis are applications which allow the user to gain and improve their skills in the field of stereometry.
Stereometry (or solid geometry) is geometry in three-dimensional space.
Stereometry is concerned with the measurement of surface areas and volumes of solid figures, such as polyhedrons, spheres, cyllinders and cones.

\section{Objective of the thesis}

The objective of the thesis is to design and implement a desktop application which allows the user to solve solid geometry tasks,
consequently expanding their skills and knowledge in this field.
After solving a task incorrectly, the user will receive a tip, to help them solve the task successfully.
A database containing all mathematical formulas, as well as tips related to the tasks will be accessible.
The user will have the possibility to change the task data and see the surface area and volume change dynamically.
An important part of the application is its customizability - the user should be able to tweak the settings of the application as well as each task such that they get the best experience tailored to their needs.

\section{Scope of the thesis}
\section{Short description of chapters}
\section{Clear description of contribution of the thesis's author}
