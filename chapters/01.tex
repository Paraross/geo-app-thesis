\chapter{Introduction}

\section{Introduction into the problem domain}

With the development of technology, education changes from classical approaches to ones using tools and methods previously unavailable.
As computers become more accessible and widespread, schools introduce more digital ways of teaching students.
One of the goals of educational applications is to aid teachers in conveing knowledge more effectively than using classical methods.
A well-designed educational application should be able to provide all the tools necessary to teach students a certain subject.
It may also be installed on the student's personal computer, allowing them to study and practice efficiently even outside of school.

The thesis is concerned with applications which allow the user to gain and improve their skills in the field of stereometry.
Stereometry (or solid geometry) is geometry in three-dimensional space.
Stereometry is concerned with the measurement of surface areas and volumes of solid figures, such as polyhedrons, spheres, cyllinders and cones.

\section{Objective and scope of the thesis}

Learning stereometry may prove difficult to a student without adequate tools.
There are several tools to learn stereometry without the assistance of a computer application. These include:
\begin{description}
    \item[Blackboard and chalk] Offer only static representations of figures which are difficult to draw precisely. The drawings are usually only wireframes and do not have shading. This might make it hard to visualize what the drawing is meant to represent. Drawing the figures is also time-consuming.
    \item[Printed textbooks] Include shading, but are still limited to static images which can be viewed from a single angle.
    \item[Physical models] Present a true three-dimesional representation which can be rotated. The dimensions are constant and not every figure might be available as a physical model.
\end{description}

The objective of the thesis is to design and implement a desktop application which helps the user learn stereometry.

Nowadays, web applications are more popular than desktop applications. Despite that, desktop applications:
\begin{itemize}
    \item do not require an internet connection after installation, while web applications require constant connection to the server hosting the service,
    \item are faster and more responsive as no data needs to be transfered between the client computer and a remote server,
    \item give the user control over software updates - the user may use an older version if desired.
\end{itemize}

The application will allow the user to solve solid geometry tasks and create polyhedrons from vertices, consequently expanding their skills and knowledge in this field.
Its target group is students aged 12 to 15.
The application can be used as a didactic tool in the classroom to help the teacher in teaching stereometry.
It can also be used at home by the student to practice stereometry outside the classroom.
The application will feature tasks.
A task will contain one or more three-dimensional figures.
Aa task will be split into several steps.
Each step will require the user to  calculate one portion of the problem described by the task.
After providing and incorrect answer, the user will receive a tip, to help them solve the task successfully.
A module allowing the user to explore the tasks and view the correct answers will be available.
In this module, the user will have the possibility to change the task data and see how the shapes change dynamically.
An important part of the application is its customizability - the user should be able to tweak the settings of the application such that they get the best experience tailored to their needs.

The application will be built using the Godot game engine. While designed primarily with video games in mind, it can be used to create other types of software. The application requires a modern graphical user interface (GUI) and 3D rendering capabilites. Godot provides a suite of tools which simplify working with both of those systems. Other solutions like Flutter\footnote{\url{https://flutter.dev}} and Electron\footnote{\url{https://www.electronjs.org}} were considered but ultimately rejected. They provide a wide range of tools for developing desktop applications but do not have first-class 3D rendering support which would make developing the application more difficult.

\section{Short description of chapters}

\begin{itemize}
    \item The second chapter analyses the problem and presents already existing solutions.
    \item The third chapter lists the functional and nonfuctional requirements,
        describes the use cases and presents the tools used in the project's development.
    \item The fourth chapter serves as the user manual.
        It specifies the sytem requirements and provides instructions on how to run and use the application.
    \item The fifth chapter is focused on the implementation of the project. It explains the architecture of the application.
    % TODO:
    \item The sixth chapter \textbf{TODO}.
    \item The seventh chapter concludes the thesis.
        It compares the achieved results to the objectives of the thesis and explores prospects of the project's further development.
\end{itemize}
