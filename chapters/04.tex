\chapter{External specification}

\section{System requirements}

A desktop computer or a laptop is required the application.
The application can run on the Windows, macOS and Linux operating systems.
The application requires approximately 100 MB of storage space.
The application was mainly tested on a laptop device with the following software and hardware configuration:

\bigskip

\begin{tabularx}{0.9\linewidth}{l X}
    CPU & AMD Ryzen 7 5800H \\
    GPU & NVIDIA GeForce RTX 3050 \\
    RAM & 16 GB \\
    Operating system & Windows 11 \\
\end{tabularx}

\bigskip

The application performed smoothly on the device.
Due to limited access to other hardware, the application could not be tested on low-end devices to establish minimum system requirements accurately.
The Godot Engine documentation lists the following as the minimum system requirements:\cite{godotspecsdocs}

\bigskip

\begin{tabularx}{0.9\linewidth}{l X}
    CPU & x86\_32 CPU with SSE2 support, x86\_64 CPU with SSE4.2 support, ARMv8 CPU \\
    GPU & Integrated graphics with full Vulkan 1.0 support, Metal 3 support (macOS) or Direct3D 12 (12\_0 feature level) support (Windows) \\
    RAM & 2 GB \\
    Operating system & Windows 10, macOS 10.15, Linux distribution released after 2018
\end{tabularx}

\section{Installation}

The application does not require installation on any supported platform.
It is launched directly from the appropriate executable.

The application was tested only on the Windows operating system.
As such, the instructions for other operating systems may contain errors.

\subsection{Windows}

For the Windows operating system, the application is provided as a single executable file.
To start the application, the user should run the \textbf{GeoApp.exe} executable.

Depending on system security settings, Windows may display a warning about an unknown publisher.
In this case, the user can confirm the warning to start the application.
The executable is compiled for the x86-64 architecture.

\subsection{Linux}

For the Linux operating system, the application is provided as a single executable file.
The user should ensure that the file has execution permissions.
This can be done through the file manager or by executing the following command in the terminal:

\begin{verbatim}
chmod +x GeoApp.x86_64
\end{verbatim}

When the file has execution permissions, the application can be started by running the \textbf{GeoApp.x86\_64} executable.
The executable is compiled for the x86-64 architecture.

\subsection{macOS}

On the macOS operating system, the application is provided in a \texttt{.zip} archive containing an application bundle.
First, the user should extract the provided \textbf{GeoApp.zip} file.
The user may then optionally move the extracted \textbf{GeoApp.app} file to the \textbf{Applications} folder.
To start the application, the user should run the \textbf{GeoApp.app} application.

Since the application is not digitally signed, a security warning may appear when the application is started for the first time.
When this happens, the user can start the application by right-clicking the \textbf{GeoApp.app} file, selecting \textbf{Open} and confirming the dialog.
Alternatively, the application can be allowed in the system settings under \textbf{Privacy \& Security}.

\section{User manual}

The application consists of several modules, namely \textit{Task exploration}, \textit{Task solving} and the \textit{Playground}.
Each module provides different ways for the user to learn stereometry.

\subsection{Main menu}

Upon starting the application, the user sees the main menu.
The main menu is a central hub from which the rest of the application can be accessed.
The main menu contains several button used to navigate the application.

\begin{itemize}
    \item The \textit{Explore tasks} button takes the user to the \textit{Task exploration} module.
    \item The \textit{Solve tasks} button takes the user to the \textit{Task solving} module.
    \item The \textit{Playground} button takes the user to the \textit{Playground} module.
    \item The \textit{Settings} button takes the user to the settings.
    \item The \textit{Quit} button closes the application.
\end{itemize}

\begin{figure}
    \centering
    \includegraphics[width=\textwidth]{graf/main_menu}
    \caption{The main menu of the application.}
\end{figure}

\subsection{Task exploration}

The purpose of the \textit{Task exploration} module is to teach the user about the tasks available in the application.
Ordered from left to right, the user interface features:

\begin{itemize}
    \item A list of all task, grouped by difficulty. The tasks can also be filtered by their name.
    \item All data describing the task.
        Each parameter can be changed.
        All UI (user interface) elements will update dynamically when a parameter is changed.
    \item The description of the task.
    \item A list of all of the steps required to solve the tasks.
        For each step, there is:
        \begin{itemize}
            \item The description of what should be calculated for the step.
            \item A help button. The button shows a hint for its respective step when pressed.
            \item A number control which displays the correct answer for its respective step.
        \end{itemize}
    \item A view of the selected task.
\end{itemize}

\begin{figure}
    \centering
    \includegraphics[width=\textwidth]{graf/explore_tasks}
    \caption{The \textit{Task exploration} module.}
\end{figure}

\subsection{Task solving}

The purpose of the \textit{Task solving} module is to test the user's knowledge and skills gained in the \textit{Task exploration} module.
The \textit{Task solving} module is split into two parts.

The first part is task selection.
Here, the user selects the task they wish to solve.
To select a task, the user must first select a difficulty.
When a difficulty is selected, all available tasks with this difficulty are shown.
Then, the user should select their desired task and press the \textbf{Start} button.
The \textbf{Start} button can be pressed only when a task is selected.

\begin{figure}
    \centering
    \includegraphics[width=\textwidth]{graf/task_filter}
    \caption{Task selection.}
\end{figure}

After pressing the \textbf{Start} button, the user is transported to the main part of the \textit{Task solving} module.

The parameter values of the task are randomly picked from a predefined range.
The range is defined on a per-parameter basis.
The precision (the number of decimal digits of the value) is also defined on a per-parameter basis.

Inside the main part of the \textit{Task solving} module, the user can find:

\begin{itemize}
    \item The view of the current task.
    \item The description of the task.
    \item The step navigation buttons.
    \item The \textit{step container}.
    \item The \textbf{Check answer} button.
    \item The \textbf{Get new task} button.
\end{itemize}

% todo: step nav

The \textit{step container} is the part of the UI responsible for displaying the task's steps.
The step container looks similar to the one present in the \textit{Task exploration} module.
Despite that, the two differ in several aspects.

First, the user can interact with the number controls.
The number controls are where the user puts their solutions to the task.
The user may edit the value only of the current step's number control.
If the user enters the correct value and presses the \textbf{Check answer} button, a green checkmark symbol appears to the right side of the number control.
If the user enters an incorrect value, a red X symbol appears instead.

Second, only steps up to the current step are visible.
The navigation buttons are used to move between steps.
The \textbf{< Prev} button moves to the previous step.
The \textbf{Next >} button moves to the next step.
The buttons are active only when the user is able to change the current step.
The \textbf{< Prev} button is inactive when the user is on the first step of the task.
The \textbf{Next >} button is inactive when the user is on the last step of the task or the answer for the current step is incorrect.

When the user enters the correct answer for the last step of the task, the task is completed.
When a task is completed, a label with the text "Task complete!" appears under the \textbf{Check answer} button.

The \textbf{Get new task} button resets the current task and provides the user with a new one with randomized values.
Returning to the task selection screen and selecting the same task achieves the same goal.

\begin{figure}
    \centering
    \includegraphics[width=\textwidth]{graf/task}
    \caption{The default state of the task solving screen.}
\end{figure}

\begin{figure}
    \centering
    \includegraphics[width=\textwidth]{graf/task_solved}
    \caption{The task solving screen after the current task is completed.}
\end{figure}

\subsection{Playground}

The playground allows the user to create an arbitrary polyhedron and inspect its properties.
The playground starts empty and the user can add vertices from which the polyhedron is created.
The vertices can be added in several ways.

\subsubsection{Adding vertices}

The easiest way is to load a predefined polyhedron.
To do this, the user should select one of the predefined polyhedrons next to the "Load a polyhedron" label.
Next, they should press the \textbf{Load} button to the right.
This loads a predefined set of vertices.

Another way to load a set of vertices is to load them from a file.
To do this, the user should press the \textbf{Load} button next to the "Load a polyhedron from a file" label.
A system file dialog should appear, expecting the user to select a file with the \texttt{.poly} extension.
No example files are provided by default.
A detailed explanation of \texttt{.poly} files is provided in section \ref{poly_files}. % TODO: numer Playground jest

% TODO: przepisac

% The last way to add a vertex is to use the \textbf{Add new vertex} button next to the "Vertices" label.
% This adds a new vertex in the origin of the coordinate system.
% When adding vertices this way, errors may appear.
% Errors appear when a polyhedron cannot be created from the provided vertices.
% Errors are explained in detail in section \ref{vertex_errors}.

% \subsubsection{Polyhedron synchronization} \label{poly_sync}
%
% The polyhedron can be in two states - synchronized or unsychronized.
% The polyhedron is synchronized when its vertices are equal to the vertices in the UI.
% Otherwise, is is unsychronized.
%
% The polyhedron becomes unsychronized when:
% a vertex is removed.
%
% \subsubsection{Incorrect vertices} \label{vertex_errors}
%
% There are two kinds of errors that may appear when editing vertices.
% When any error related to the vertices is present, a red exclamation mark appears next to the "Vertices" label.
%
% The first kind of error is the duplicate vertex error.
% It appears when two vertices are equal.
% To resolve the issue, the user should make sure that each vertex is unique.
%
% The other kind of error is the "not in polyhedron" error.

% TODO: niewlasciwe przy ladowaniu z pliku

\begin{figure}
    \centering
    \includegraphics[width=\textwidth]{graf/playground}
    \caption{The default state of the playround module.}
\end{figure}

\subsubsection{Polyhedron files} \label{poly_files}

\subsection{Settings}

% \begin{tabular}{ |c|c|c| } 
%  \hline
%  cell1 & cell2 & cell3 \\ 
%  cell4 & cell5 & cell6 \\ 
%  cell7 & cell8 & cell9 \\ 
%  \hline
% \end{tabular}

% \begin{itemize}
% \item hardware and software requirements
% \item installation procedure
% \item activation procedure
% \item types of users
% \item user manual
% \item system administration
% \item security issues
% \item example of usage
% \item working scenarios (with screenshots or output files)
% \end{itemize}
 
% \begin{figure}
% \centering
% \begin{tikzpicture}
% \begin{axis}[
%     y tick label style={
%         /pgf/number format/.cd,
%             fixed,    
%             fixed zerofill, % 1.0 zamiast 1
%             precision=1,
%         /tikz/.cd
%     },
%     x tick label style={
%         /pgf/number format/.cd,
%             fixed,
%             fixed zerofill,
%             precision=2,
%         /tikz/.cd
%     }
% ]
% \addplot [domain=0.0:0.1] {rnd};
% \end{axis} 
% \end{tikzpicture}
% \caption{Figure caption (below the figure).}
% \label{fig:2}
% \end{figure}

%%%%%%%%%%%%%%%%%%%%%
% FIGURE FROM FILE
%
%\begin{figure}
%\centering
%\includegraphics[width=0.5\textwidth]{./graf/politechnika_sl_logo_bw_pion_en.pdf}
%\caption{Caption of a figure is always below the figure.}
%\label{fig:label}
%\end{figure}
%Fig. \ref{fig:label} presents …
%%%%%%%%%%%%%%%%%%%%%
%
%%%%%%%%%%%%%%%%%%%%
%% SUBFIGURES
%
%\begin{figure}
%\centering
%\begin{subfigure}{0.4\textwidth}
%    \includegraphics[width=\textwidth]{./graf/politechnika_sl_logo_bw_pion_en.pdf}
%    \caption{Upper left figure.}
%    \label{fig:upper-left}
%\end{subfigure}
%\hfill
%\begin{subfigure}{0.4\textwidth}
%    \includegraphics[width=\textwidth]{./graf/politechnika_sl_logo_bw_pion_en.pdf}
%    \caption{Upper right figure.}
%    \label{fig:upper-right}
%\end{subfigure}
%
%\begin{subfigure}{0.4\textwidth}
%    \includegraphics[width=\textwidth]{./graf/politechnika_sl_logo_bw_pion_en.pdf}
%    \caption{Lower left figure.}
%    \label{fig:lower-left}
%\end{subfigure}
%\hfill
%\begin{subfigure}{0.4\textwidth}
%    \includegraphics[width=\textwidth]{./graf/politechnika_sl_logo_bw_pion_en.pdf}
%    \caption{Lower right figure.}
%    \label{fig:lower-right}
%\end{subfigure}
%        
%\caption{Common caption for all subfigures.}
%\label{fig:subfigures}
%\end{figure}
%Fig. \ref{fig:subfigures} presents very important information, eg. Fig. \ref{fig:upper-right} is an upper right subfigure.
%%%%%%%%%%%%%%%%%%%%%
